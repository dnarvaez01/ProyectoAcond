\section{Conclusiones}
Se logró cumplir con las actividades propuestas en el cronograma, logrando medir y analizar el estado actual de las
salas, crear una propuesta de mejora en cada uno de los recintos y realizar un presupuesto de estas soluciones.\\
A raíz de los resultados de las mediciones, se logró identificar si estos son adecuados acústicamente para el uso 
de cada uno de estos recintos. Por un lado, el ruido presente en los salones es el problema que predomina, 
incumpliendo con las curvas NC recomendadas para su uso.\\
En el caso del tiempo de reverberación en las salas de reuniones según la norma DIN$18041$ estos no cumplen con 
valores acordes a su uso. Lo cual con la propuesta dada en este proyecto se logró cumplir con este criterio, además
de mejorar la claridad y la inteligibilidad de la palabra en ambas salas.\\
En el caso del tiempo de reverberación de la sala de ensayo, los valores eran demasiado bajos con la implementación
que se realizó por parte de la administración, los valores obtenidos no eran aptos para que el sonido resonara y 
cree intimidad musical, que es lo que se busca en estos recintos. Con la solución de retirar este material, 
aumentaría el tiempo de reverberación, mejorando la claridad y la definición del sonido de los instrumentos.\\
