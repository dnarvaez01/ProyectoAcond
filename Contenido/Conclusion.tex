\section{Conclusiones parciales}
A raíz de los resultados de las mediciones, se logró identificar si estos son adecuados acústicamente para el uso de cada uno de estos recintos. Por un lado, el ruido presente en los salones es el problema que predomina, el cual puede confundirse con un mal acondicionamiento de las salas, al momento de dar un diagnóstico del estado de los recintos.

Por otro lado, el tiempo de reverberación es perjudicial al momento de estudiar un recinto para el habla, como lo son las salas de conferencia. Sin embargo, en un salón para practicar musica clásica, este se puede utilizar a favor para sacar el mayor potencial del sonido de los instrumentos. 

En la situación actual de las salas de reuniones el tiempo de reverberación y claridad están dentro de lo recomendado, por lo que estos no necesitan de una gran intervención en el acondicionamiento, el problema que radica en estos salones es el ruido, producido probablemente por la ubicación geográfica cercana a una avenida con un transito vehicular. Para el caso de la sala de ensayo, los valores de tiempo de reverberación son bajos con respecto a lo recomendado, esto probablemente a la intervención que se le realizó previamente al recinto.

En trabajos próximos se espera mejorar el estado de la sala de ensayo y dar una posible solución al problema de ruido producido por el transito vehicular en los tres recintos.
