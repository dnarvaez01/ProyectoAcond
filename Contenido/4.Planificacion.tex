\section{Planificación}
\subsection{Plan de trabajo}
Para elaborar el proyecto es importante la realización de actividades las cuales se ven descritas en la tabla \ref{tab: plan de trabajo} con sus respectivos resultados esperados para un trabajo óptimo.
\begin{table}[H]
\centering
\caption{Plan de trabajo}
\label{tab: plan de trabajo}
\resizebox{\textwidth}{!}{%
\begin{tabular}{|c|l|l|l|}
\hline
\textbf{Actividad}    & \multicolumn{1}{c|}{\textbf{Descripción}}                                                                                                             & \multicolumn{1}{c|}{\textbf{Resultado esperado}}                                                                                                         & \multicolumn{1}{c|}{\textbf{Indicador de cumplimiento}}                                                                                        \\ \hline
Planificación         & \begin{tabular}[c]{@{}l@{}}Establecer contacto con el \\ establecimiento y \\ planificación de actividades\end{tabular}                               & Organización del proyecto                                                                                                                                & Carta Gantt y puntos de medición                                                                                                               \\ \hline
Mediciones            & \begin{tabular}[c]{@{}l@{}}Realizar mediciones de \\ dimensión, ruido de fondo \\ y tiempo de reverberación \\ de las salas\end{tabular}              & \begin{tabular}[c]{@{}l@{}}Mediciones de respuesta al \\ impulso de las salas, \\ geometría y volumen\end{tabular}                                       & \begin{tabular}[c]{@{}l@{}}Datos de características físicas y \\ acústicas del recinto\end{tabular}                                            \\ \hline
Modelación            & \begin{tabular}[c]{@{}l@{}}Modelar las salas en \\ software CAD\end{tabular}                                                                          & \begin{tabular}[c]{@{}l@{}}Modelo en software EASE\\  del recinto\end{tabular}                                                                           & \begin{tabular}[c]{@{}l@{}}Archivo de software EASE con \\ modelo del recinto\end{tabular}                                                     \\ \hline
Análisis de datos     & \begin{tabular}[c]{@{}l@{}}Procesar y tabular los datos\\ obtenidos de las mediciones\end{tabular}                                                    & \begin{tabular}[c]{@{}l@{}}Tabla comparativa con \\ recomendaciones\end{tabular}                                                                         & \begin{tabular}[c]{@{}l@{}}Archivo CSV con los datos \\ procesados de las mediciones\end{tabular}                                              \\ \hline
Diseño de propuesta   & \begin{tabular}[c]{@{}l@{}}Proponer un diseño de \\ acondicionamiento acústico \\ acorde con recomendaciones \\ para las salas\end{tabular}           & \begin{tabular}[c]{@{}l@{}}Determinar una solución de\\  acondicionamiento acústico\\ óptimo para los usos de las \\ salas\end{tabular}                  & \begin{tabular}[c]{@{}l@{}}Documentación del detalle y \\ archivo de software EASE con la \\ implementación de la propuesta\end{tabular}       \\ \hline
Análisis de propuesta & \begin{tabular}[c]{@{}l@{}}Evaluar el acondicionamiento \\ acústico a través \\ de software EASE\end{tabular}                                         & \begin{tabular}[c]{@{}l@{}}Comprobar el mejoramiento \\ de las características acústicas\\ de cada sala\end{tabular}                                     & \begin{tabular}[c]{@{}l@{}}Tabla comparativa de los parámetros \\ de las salas acondicionadas y \\ recomendaciones bibliográficas\end{tabular} \\ \hline
Presupuesto de diseño & \begin{tabular}[c]{@{}l@{}}Presentar una cotización de \\ los elementos y materiales\\  propuestos para el \\ acondicionamiento acústico\end{tabular} & \begin{tabular}[c]{@{}l@{}}Listado de los elementos \\ requeridos para el \\ acondicionamiento en el\\ mercado y sus respectivos \\ valores\end{tabular} & \begin{tabular}[c]{@{}l@{}}Presupuesto de los elementos que \\ componen la propuesta de \\ acondicionamiento acústico\end{tabular}             \\ \hline
Redacción de informe  & \begin{tabular}[c]{@{}l@{}}Evidenciar el desarrollo del \\ proyecto mediante avances \\ progresivos\end{tabular}                                      & \begin{tabular}[c]{@{}l@{}}Plasmar los resultados del \\ trabajo realizado en el \\ proyecto\end{tabular}                                                & Informe final del proyecto                                                                                                                     \\ \hline
Elaboración de póster & \begin{tabular}[c]{@{}l@{}}Sintetizar la información \\ presente en el informe en \\ formato póster\end{tabular}                                      & Resumen general del proyecto                                                                                                                             & Póster                                                                                                                                         \\ \hline
\end{tabular}%
}
\end{table}


\newpage

\subsection{Cronograma}
\noindent A continuación, en la tabla \ref{tab: carta gantt} se puede observar la planificación respectiva del proyecto, partir de las actividades planteadas en la tabla \ref{tab: plan de trabajo}. Esta planificación se plantea por semanas, desde la semana del $21$ de agosto hasta la semana del $2$ de diciembre del presente año. Los números señalan la cantidad de horas que se emplearan en realizar cada actividad. Las semanas destacadas son las que corresponden a las entregas de avances del proyecto.
\begin{table}[H]
\centering
\caption{Carta Gantt del proyecto}
\label{tab: carta gantt}
\resizebox{16cm}{!} {
\begin{tabular}{|l|llllllllllllllll|}
\hline
 & \multicolumn{16}{c|}{Semanas} \\ \cline{2-17} 
\multirow{-2}{*}{Actividad} & $1$ & \cellcolor[HTML]{FABF8F}$2$ & $3$ & {\cellcolor[HTML]{FABF8F}$4$} & $5$ & $6$ & $7$ & {\cellcolor[HTML]{FABF8F}$8$} & $9$ & $10$ & \cellcolor[HTML]{FABF8F}$11$ & \multicolumn{1}{l}{$12$} & $13$ & $14$ & \multicolumn{1}{|l|}{\cellcolor[HTML]{FABF8F}$15$} & \cellcolor[HTML]{FABF8F}$16$ \\ \hline
Planificación & \cellcolor[HTML]{4F81BD}$3$ & \cellcolor[HTML]{4F81BD}$4$ &  &  &  &  &  &  &  &  &  &  &  &  & \multicolumn{1}{|l|}{} &  \\
Mediciones &  & \cellcolor[HTML]{4F81BD}$6$ & {\cellcolor[HTML]{4F81BD}$3$} &  &  &  &  &  &  &  &  &  &  &  & \multicolumn{1}{|l|}{} &  \\
Modelación &  &  & {\cellcolor[HTML]{4F81BD}$6$} & {\cellcolor[HTML]{4F81BD}$6$} & \cellcolor[HTML]{4F81BD}$6$ &  &  &  &  &  &  &  &  &  & \multicolumn{1}{|l|}{} &  \\
Análisis de datos &  &  &  & {\cellcolor[HTML]{C0504D}$2$} & \cellcolor[HTML]{C0504D}$4$ &  &  &   &  &  &  &  &  &  & \multicolumn{1}{|l|}{} &  \\
Diseño de propuesta &  &  &  &  & \cellcolor[HTML]{C0504D}$4$ & \cellcolor[HTML]{C0504D}$4$ & \cellcolor[HTML]{C0504D}$4$ & {\cellcolor[HTML]{C0504D}$4$} & \cellcolor[HTML]{C0504D}$4$ & \cellcolor[HTML]{C0504D}$4$ &  &  &  &  & \multicolumn{1}{|l|}{} &  \\
Análisis de propuesta &  &  &  &  & \cellcolor[HTML]{C0504D}$2$ & \cellcolor[HTML]{C0504D}$2$ & \cellcolor[HTML]{C0504D}$2$ & {\cellcolor[HTML]{C0504D}$2$} & \cellcolor[HTML]{C0504D}$2$ & \cellcolor[HTML]{C0504D}$2$ &  &  &  &  & \multicolumn{1}{|l|}{} &  \\
Presupuesto de diseño &  &  &  &  &  &  &  &   & \cellcolor[HTML]{C0504D}$2$ & \cellcolor[HTML]{C0504D}$2$ & \cellcolor[HTML]{C0504D}$2$ & \cellcolor[HTML]{C0504D}$2$ &  &  & \multicolumn{1}{|l|}{} &  \\
Redacción de informe & \cellcolor[HTML]{9BBB59}$2$ & \cellcolor[HTML]{9BBB59}$2$ & {\cellcolor[HTML]{9BBB59}$2$} & {\cellcolor[HTML]{9BBB59}$2$} &  & \cellcolor[HTML]{9BBB59}$2$ & {\cellcolor[HTML]{9BBB59}$2$} & {\cellcolor[HTML]{9BBB59}$2$} &  & \cellcolor[HTML]{9BBB59}
$2$ & \cellcolor[HTML]{9BBB59}$2$ & \cellcolor[HTML]{9BBB59}$2$ & {\cellcolor[HTML]{9BBB59}$2$} & \cellcolor[HTML]{9BBB59}$2$ & \multicolumn{1}{|l|}{\cellcolor[HTML]{9BBB59}$2$} &  \\
Elaboración de poster &  &  &  &  &  &  &  &  &  &  &  &  &  &  & \multicolumn{1}{|l|}{\cellcolor[HTML]{538DD5}$2$} & \cellcolor[HTML]{538DD5}$2$ \\ \hline
\end{tabular}%
}
\end{table}
\noindent Actualmente el proyecto se encuentra en la semana $15$, en la cual se hace la última entrega de informe, teniendo ya una solución y un presupuesto de este.