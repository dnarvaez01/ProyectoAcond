\section{Introducción}
\noindent El acondicionamiento acústico consiste en la definición de las formas y revestimientos de las superficies interiores de un recinto con objeto de conseguir las condiciones acústicas más adecuadas para el tipo de actividad a la que se haya previsto destinarlo.\cite{carrion1990diseno} \\
El presente informe tiene como objetivo analizar y proponer soluciones para el acondicionamiento acústico de dos salas de reuniones y una sala de ensayo de orquesta. \\

\noindent
La importancia del acondicionamiento acústico radica en su capacidad para mejorar la experiencia auditiva de las personas que utilizan estos espacios.
En el caso de las salas de reuniones, es imprescindible una óptima inteligibilidad de la palabra ya que la comprensión del mensaje oral es de suma importancia. \cite{carrion1990diseno}
En el contexto de una sala de ensayo de orquesta, el acondicionamiento acústico es crucial para lograr una reproducción fiel y equilibrada de los instrumentos musicales, permitiendo a los músicos escucharse entre sí y trabajar en conjunto de manera óptima. \\
\noindent
Para llevar a cabo este proyecto, se utilizará normativas que indican el procedimiento de medición tiempo de reverberación y cálculo de parámetros acústicos como la ISO $3382-2$ \cite{ISO3382-2}.

\section{Objetivos}
\begin{itemize}
    \item \textbf{Objetivo General:} \\
    Proponer un acondicionamiento acústico para dos salas de reuniones y una sala de ensayo de orquesta, en el Centro de Extensión Campus los Canelos, UACh.

    \item \textbf{Objetivos Específicos:}
    \begin{itemize}
        \item Caracterizar acústicamente las salas.
        \item Analizar los parámetros acústicos de las salas.
        \item Generar una propuesta de diseño de acondicionamiento acústico para optimizar el sonido de las salas.
        \item Comparar los resultados mediante recomendaciones internacionales. 
        \item Estimar presupuestos del acondicionamiento acústico para cada sala.

    \end{itemize}
\end{itemize}