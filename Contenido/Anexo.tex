\section{Anexo}
\subsection{Presupuesto} 
\noindent A continuación, se muestra el presupuesto a partir de las horas invertidas descritas en la planificación, dando un total de \$ $3.842.227$ pesos. 
\begin{table}[H]
\centering
\caption{Presupuesto del proyecto}
\label{tab: presupuesto}
\resizebox{\columnwidth}{!}{%
\begin{tabular}{clc|c|c|c|}
\hline
\multicolumn{1}{|l|}{} & \multicolumn{1}{l|}{\textbf{Actividades}} & \textbf{Horas} & \textbf{UF/hr} & \textbf{UF*} & \textbf{CLP} \\ \hline
\multicolumn{1}{|c|}{\multirow{8}{*}{\textbf{Costo ingeniería}}} & \multicolumn{1}{l|}{Planificación} & $7$ & $0.6$ & $4.2$ & $151.687$ \\
\multicolumn{1}{|c|}{} & \multicolumn{1}{l|}{Mediciones} & $9$ & $1$ & $9$ & $325.044 $\\
\multicolumn{1}{|c|}{} & \multicolumn{1}{l|}{Modelación} & 1$8 $& $0.8$ & $14.4$ & $520.070$ \\
\multicolumn{1}{|c|}{} & \multicolumn{1}{l|}{Análisis de datos} & $6$ & $1$ & $6$ & $216.696$ \\
\multicolumn{1}{|c|}{} & \multicolumn{1}{l|}{Diseño de propuesta} & $24$ & $1$ & $24$ & $866.784$ \\
\multicolumn{1}{|c|}{} & \multicolumn{1}{l|}{Análisis de propuesta} & $12$ & $0.8$ & $9.6$ & $346.713$ \\
\multicolumn{1}{|c|}{} & \multicolumn{1}{l|}{Cotización de propuesta} & $8$ & $0.6$ & $4.8$ & $173.356$ \\
\multicolumn{1}{|c|}{} & \multicolumn{1}{l|}{Redacción de informe} & $26$ & $0.6$ & $15.6$ & $563.409$ \\ \hline
\multicolumn{1}{|c|}{\multirow{2}{*}{\textbf{Costo operacional}}} & \multicolumn{1}{l|}{Arriendo de equipos} &  &  & $1.7$ & $60.000$ \\
\multicolumn{1}{|c|}{} & \multicolumn{1}{l|}{Traslados} &  &  & $0.1$ & $5.000$ \\ \hline
\multicolumn{1}{l|}{} & \multicolumn{1}{l|}{\textbf{Total}} & $110$ &  & $89.3$ & $3.228.762$ \\ \cline{2-6} 
\multicolumn{1}{l}{} &  &  & IVA ($19$\%) & $17$ & $613.465$ \\ \cline{4-6} 
\multicolumn{1}{l}{} &  &  & Total (con IVA) & $106.3$ & $3.842.227$ \\ \cline{4-6} 
\end{tabular}%
}
\end{table}

*\textit{Considerando el valor en UF correspondiente a la fecha del $28$ de agosto de $2023$ es de \$ $36.116$ CLP.}


\subsection{Marco teórico}
\subsubsection{Ruido de fondo (RdF)}
El ruido de fondo es el sonido presente en una sala cuando no hay actividad o no está presente la fuente a evaluar. Este ruido puede ser causado por sistemas de climatización, instalaciones eléctricas o hidráulicas, e incluso ruido externo como el tráfico.

\begin{itemize}

    \item \textbf{Curvas NC}: son curvas de referencia que representan los límites de ruido aceptables en un espacio según su uso. Un recinto se considera que cumple una especificación NC determinada (por ejemplo, NC-$15$, NC-$20$, etc.) cuando los niveles de ruido de fondo, medidos en diferentes bandas de frecuencia, están por debajo de la curva NC correspondiente para todas las frecuencias entre $63$ Hz y $8$ kHz (ver figura \ref{fig:curvas NC}).\cite{carrion1990diseno}
    \begin{figure}[H]
        \centering
        \includegraphics[width=8cm]{Imagenes/MarcoTeorico/CurvaNC.png}
        \caption{Curvas NC (``Noise Criteria'')}
        \label{fig:curvas NC}
    \end{figure}
    En la tabla \ref{tab:recomendaciones Curva NC} se muestran las curvas NC recomendadas para diferentes tipos de recinto, junto con su equivalencia en dBA.
    \begin{table}[H]
        \centering
        \caption{Curvas NC recomendadas y niveles de ruido de fondo equivalentes en dBA }
        \label{tab:recomendaciones Curva NC}
        \begin{tabular}{|l|c|}
        \hline
        \textbf{Tipos de recinto} & \textbf{\begin{tabular}[c]{@{}l@{}}Curva NC\\ recomendada\end{tabular}} \\ \hline
        Fábricas para ingeniería pesada & $55$ - $75$ \\ \hline
        Fábricas para ingeniería ligera & $45$ - $65$ \\ \hline
        Cocinas industriales & $40$ - $50$ \\ \hline
        Recintos deportivos y piscinas & $35$ - $50$ \\ \hline
        Grandes almacenes y tiendas & $35$ - $45$\\ \hline
        Restaurantes, bares, cafeterías y cafeterías privadas & $35$ - $50$  \\ \hline
        Oficinas mecanizadas & $40$ - $50$  \\ \hline
        Oficinas generales & $35$ - $45$  \\ \hline
        Despachos, bibliotecas, salas de justicia y aulas & $30$ - $35$ \\ \hline
        viviendas, dormitorios & $25$ - $35$ \\ \hline
        Salas de hospitales y quirófanos & $25$ - $35$  \\ \hline
        Cines & $30$ - $35$\\ \hline
        Teatros, salas de juntas, iglesias & $25$ - $30$ \\ \hline
        Salas de conciertos y teatros de ópera & $20$ - $25$ \\ \hline
        Estudios de registro y reproducción sonora & $15$ - $20$ \\ \hline
        \end{tabular}
    \end{table}

    \item \textbf{Curvas NR}: Las curvas NR son estándares que muestran los niveles de ruido de fondo aceptables en recintos. Se dividen en números NR (como NR-15, NR-25, NR-35), que indican el nivel máximo de ruido permitido en decibelios (dB) según el propósito del espacio.
En la figura \ref{NR}, se muestran las curvas NR de evaluación de ruido y en la tabla \ref{tabla} , figuran los valores recomendados del índice de NR para diferentes locales.\cite{Recuero}

\begin{figure}[H]
    \centering \includegraphics[scale=0.6]{Imagenes/MarcoTeorico/NR.jpg}
    \caption{Curvas NR}
    \label{NR}
\end{figure}

\begin{table}[H]
\centering
\caption{Valores recomendados de índice NR para distintos locales.}
\label{Valores recomendados de NR}
\resizebox{15 cm}{!}{%
\begin{tabular}{|l|c|}
\hline
\multicolumn{1}{|c|}{\textbf{Tipos de recintos}}                                                       & \textbf{Rangos de niveles NR} \\ \hline
Talleres                                                                                               & $60$ - $70$                       \\ \hline
Oficinas mecanizadas                                                                                   & $50$ - $55$                       \\ \hline
Gimnasios, piscinas, salas de deporte, pasillos                                                                  & $40$ - $50$                       \\ \hline
Restaurantes, bares y cafeterías                                                                       & $35$ - $45$                       \\ \hline
Despachos, bibliotecas, salas de justicias                                                             & $30$ - $40$                       \\ \hline
\begin{tabular}[c]{@{}l@{}}Cines, hospitales, iglesias, pequeñas salas y aulas de conferencias \end{tabular} & $25$ - $35$                       \\ \hline
\begin{tabular}[c]{@{}l@{}}Estudio de televisión, grandes salas de conferencias\end{tabular} & $20$ - $30$                       \\ \hline
Sala de conciertos, teatros                                                                            & $20$ - $25$                       \\ \hline
Clínicas, recintos para audiometrías                                                                   & $10$ - $20$                       \\ \hline
\end{tabular}%
}
\label{tabla}
\end{table}
    
\end{itemize}

\subsubsection{Tiempo de reverberación (TR)}
El tiempo de reverberación $T_{60}$ es el tiempo que tarda el sonido en decaer $60$ dB en un espacio cerrado. Este parámetro es importante en el diseño de un sistema de refuerzo sonoro, ya que afecta la claridad, inteligibilidad y calidad del sonido en un recinto.

\begin{itemize}
    \item $T_{30}$: Es una expresión del $T_{60}$ que mide el tiempo que demora el sonido en decaer $30$ dB. El valor $T_{30}$ se entrega ya multiplicado por el factor correcto para medir la que sería la caída por $60$ dB.
    \item Formula de Sabine: Uno de los modelos para calcular el tiempo de reverberación es la fórmula de Sabine. La cual es calculada según lo indicado en la ecuación \ref{eq:T60}. \cite{sabine1922collected}
    \begin{equation} \label{eq:T60} 
     T_{60}=0.161 \cdot \frac{V_{t}}{S_{t} \bar{\alpha}}
    \end{equation}

    donde:
    \begin{itemize}
        \item $V_t$ es el volumen total de la sala
        \item $S_t$ es la superficie total de la sala
        \item $\bar{\alpha}$ es la absorción media de la sala
\end{itemize}
    \item $RT_{mid}$: Es la media aritmética de los valores de tiempo de reverberación correspondientes a las bandas de $500$ Hz y $1$ kHz. En general, el valor más adecuado de $RT_{mid}$ depende del volumen del recinto y de la actividad a la que este destinado este. En la tabla \ref{tab:rtmid recomendado por tipo de sala} se muestran valores recomendados de $RT_{mid}$, para diferentes tipos de salas en el supuesto que estén ocupadas. \cite{carrion1990diseno}
    \begin{table}[H]
        \centering
        \begin{tabular}{|l|c|}
        \hline
             \textbf{Tipo de sala} &  $RT_{mid}$\textbf{, sala ocupada}\\ \hline
             Sala de conferencia& 
             $0.7$ - $1.0$\\
             Cine             & $1.0$ - $1.2$\\
             Sala polivalente & $1.2$ - $1.5$\\
             Teatro de ópera & $1.2$ - $1.5$\\
             Sala de conciertos (música de cámara) & $1.3$ - $1.7$\\
             Sala de conciertos (música sinfónica) & $1.8$ - $2.0$\\
             Iglesia/catedral (órgano y canto coral) & $2.0$ - $3.0$\\
             Locutorio de radio & $0.2$ - $0.4$\\ \hline
        \end{tabular}
        \caption{Valores de $RT_{mid}$ recomendados en función del tipo de sala (recintos ocupados)}
        \label{tab:rtmid recomendado por tipo de sala}
    \end{table}
    En el caso de salas de salas de conferencia/aulas el $RT_{mid}$ recomendado, considerando volúmenes entre los $100$ y $10000$ $m^3$, es entre los valores de:
    \[0.7 \leq RT_{mid} \leq 1 s \]

    \item Claridad: La claridad se refiere a cómo se distribuye la energía del sonido en un espacio. En términos simples, se mide la relación entre el sonido directo y las primeras reflexiones (llamada energía temprana) en comparación con la energía que llega después. En acústica, se utiliza el parámetro $C_{50}$ para evaluar la claridad en entornos destinados a la voz y el $C_{80}$ para la música. Estos valores se obtienen de manera logarítmica y se calculan para un rango de frecuencias entre $125$ y $4000$ Hz.\cite{carrion1990diseno}
    \begin{equation}
      C_{50}= 10 log_{10} \left[\int_0^{50 ms}[g(t)]^2 dt/\int_{50 ms}^{\infty}[g(t)]^2 dt\right]  
    \end{equation}
    \begin{equation}
      C_{80}= 10 log_{10} \left[\int_0^{80 ms}[g(t)]^2 dt/\int_{80 ms}^{\infty}[g(t)]^2 dt\right]  
    \end{equation}
    Marshall, define valores apropiados de $C_{50}$ para el habla y $C_{80}$ para distintos tipos de música, descritos en la figura \ref{fig:Recomendaciones C50 C80} \cite{marshall1994}
    \begin{figure}[H]
        \centering
        \includegraphics[scale=0.5]{Imagenes/MarcoTeorico/Recomendaciones C50-C80.png}
        \caption{Recomendaciones de claridad para el habla y música}
        \label{fig:Recomendaciones C50 C80}
    \end{figure}
    \item Definición: El parámetro llamado definición se determinó para evaluar la cantidad de energía temprana, el cual se deriva directamente de la respuesta impulso $g(t)$ en la siguiente ecuación:
    \begin{equation}
        D_{50}= \left[\int_0^{50 ms}[g(t)]^2 dt/\int_{0}^{\infty}[g(t)]^2 dt\right] 
    \end{equation}
    Ambas integrales deben incluir el sonido directo, cuya llegada al oyente determina el tiempo $t = 0$. Obviamente, D será del $100\%$ si la respuesta al impulso no contiene componentes con retardos superiores a $50$ ms. \cite{Kuttruff_2017} 

    %insertar figura de Relación entre inteligibilidad de sílabas y definición.
    
    \item STI: Es el índice de transmisión del habla el cual indica el entendimiento de la palabra y sus valores oscilan entre los $0$ y $1$, que evalúa la inteligibilidad en el recinto, siendo los valores cercanos a 0 deficientes y los valores cercanos a $1$ una inteligibilidad excelente. Según la normativa ISO $9921$ \cite{ISO9921} los valores clasificarían como se indica en la tabla \ref{tab: rango STI}. 

\begin{table}[H]
    \centering
    \caption{Rango de STI según ISO $9921$}
    \label{tab: rango STI}
    \begin{tabular}{|c|c|}
    \hline
    \textbf{Rango de inteligibilidad} & \textbf{STI} \\ \hline
    Excelente                &     $>0.75$     \\ \hline
    Bueno                    & $0.60$ - $0.75$ \\ \hline
    Razonable                & $0.45$ - $0.60$ \\ \hline
    Pobre                    & $0.30$ - $0.45$ \\ \hline
    Malo                     & $<0.3$ \\ \hline
    \end{tabular}
\end{table}
\end{itemize}

\subsubsection{Medición de coeficiente de absorción en tubo de impedancia}
Un método utilizado para determinar el coeficiente de absorción acústica de un material es a través del empleo de un tubo de impedancia, también conocido como tubo de Kundt. Este dispositivo consta de un tubo equipado con dos micrófonos dispuestos a lo largo de su longitud, junto con un altavoz ubicado en uno de sus extremos. En el extremo opuesto se sitúa el material cuya capacidad de absorción acústica se pretende medir.
A partir de la relación de la señal recibida por cada uno de los micrófonos el software ... entrega el coeficiente de reflexión ... el cual se utiliza en la ecuación... para calcular el coeficiente de absorción.
\begin{equation}
    \alpha = 1 - |r|^2 = 1 - r_{r}^2 - r_{i}^2
\end{equation}
donde,\\
$r$ es el factor de reflexión.\\
$r_{r}$ es el componente real del factor de reflexión.\\
$r_{i}$ es el componente imaginario del factor de reflexión.\\
Posteriormente a la obtención de los datos se realiza el cálculo de los valores de coeficientes de
absorción ponderado, para esto fue necesario transformar los datos obtenidos anteriormente a tercios
de octava y así calcular el coeficiente de absorción sonora práctico ($\alpha_{pi}$), que se define como el valor
del coeficiente de absorción acústica dependiente de la frecuencia, basado en mediciones por bandas
de un tercio de octava, que se obtiene mediante la siguiente fórmula:
\begin{equation}
    \alpha_{pi} = \frac{\alpha_{i1} + \alpha_{i2} + \alpha_{i3}}{3}
\end{equation}
Ahora con los coeficiente de absorción práctico, se obtienen los valores de coeficiente de absorción
ponderado $\alpha_{\omega}$, utilizando la curva de referencia de la figura ... Se irá ajustando la curva de referencia
por intervalos de $0.05$ hacia el valor calculado hasta que la suma de las desviaciones desfavorables
sea menor o igual que $0.10$, esto ocurre cuando el valor medido es menor que el valor de la curva de
referencia.

\subsection{Posiciones de micrófono y fuente en mediciones de tiempo de reverberación}
A continuación se pueden observar la distribución de las posiciones de fuente y micrófono para las mediciones de tiempo de reverberación.
\begin{figure}[H]
    \centering
    \includegraphics[scale=0.4]{Imagenes/PosicionesRT/Posiciones Sala 1.png}
    \caption{Posiciones de fuente y micrófono para sala de reunión 1}
    \label{fig: posiciones sala1}
\end{figure}

\begin{figure}[H]
    \centering
    \includegraphics[scale=0.4]{Imagenes/PosicionesRT/Posiciones Sala 2.png}
    \caption{Posiciones de fuente y micrófono para sala de reunión 2}
    \label{fig: posiciones sala2}
\end{figure}

\begin{figure}[H]
    \centering
    \includegraphics[scale=0.4]{Imagenes/PosicionesRT/Posiciones Sala OCV.png}
    \caption{Posiciones de fuente y micrófono para sala de ensayo}
    \label{fig: posiciones sala OCV}
\end{figure}

\subsection{Mediciones de tiempo de reverberación}
A continuación se pueden observar los resultados de cada posición de micrófono y fuente de tiempo de reverberación
\begin{figure}[H]
    \centering
    \includegraphics[width=12cm]{Imagenes/Resultados/T60_Sala_reunion_1.png}
    \caption{$T_{60}$ sala de reunión 1}
    \label{fig: T60 sala1}
\end{figure}

\begin{figure}[H]
    \centering
    \includegraphics[width=12cm]{Imagenes/Resultados/T60_Sala_reunion_2.png}
    \caption{$T_{60}$ sala de reunión 2}
    \label{fig: T60 sala2}
\end{figure}

\begin{figure}[H]
    \centering
    \includegraphics[width=12cm]{Imagenes/Resultados/T60_Sala_de_ensayo.png}
    \caption{$T_{60}$ sala de ensayo}
    \label{fig: T60 sala de ensayo}
\end{figure}

\subsection{Resultados curvas NR}
    \begin{figure}[H]
        \centering
        \includegraphics[width=10cm]{Imagenes/Resultados/Curvas NC-NR/NR reunion 1.png}
        \caption{Curvas NR sala de reunión 1}
        \label{fig: Curvas NR sala 1}
    \end{figure}

    \begin{figure}[H]
        \centering
        \includegraphics[width=10cm]{Imagenes/Resultados/Curvas NC-NR/NR reunion 2.png}
        \caption{Curvas NR sala de reunión 2}
        \label{fig: Curvas NR sala 2}
    \end{figure}

    \begin{figure}[H]
        \centering
        \includegraphics[width=10cm]{Imagenes/Resultados/Curvas NC-NR/NR ensayo.png}
        \caption{Curvas NR sala de ensayo}
        \label{fig: Curvas NR sala de ensayo}
    \end{figure}